\chapter{Preliminaries}

\section{Term frequency - inverse document frequency}

\emph{Term frequency - inverse document frequency (tf-idf)}
\cite{salton1988term} is a popular measure in \emph{information retrieval} that
captures the importance of a word within a given document (from a corpus).
It is computed based on two different weights:
\begin{description}
  \item[Term frequency -- tf(d, w)] measures the frequency of a word in a
  document (usually normalised by taking the square root, logarithm or more
  complex methods);
  \item[Inverted document frequency -- idf(d, w)] measures the rarity of a term
  within the corpus -- it is defined as the total number of documents divided
  by the number of documents in which the word appears (usually normalised by
  taking the logarithm).
\end{description}
The final \emph{tf-idf} value is obtained by multiplying the two weights
defined above:
\[ \text{tf-idf}(d, w) = \text{tf}(d, w) \cdot \text{idf}(d, w). \]

\section{Cosine similarity}

The \emph{cosine similarity} of two vectors is defined as the angle between
made by the two vectors.
Formally, cosine similarity is defined as:
\[\cos(\theta) = \frac{\vec{u} \cdot \vec{v}}{\|\vec{u}\| \cdot \|\vec{v}\|}\]
or if the vectors are normalized simply as the dot product between of the two
vectors:
\[\cos(\theta) = \vec{u} \cdot \vec{v}.\]

\section{Locality-sensitive hashing}

\emph{\ac{LSH}} \cite{rajaraman2012mining} is a method to put similar elements
into the same bucket with high probability. This is achieved by using hash
functions that, instead of trying to uniformly distributed the elements among
all buckets, are specifically designed to hash similar document to the same
value with high probability.
A lot of similarity measures (or distances) have an associated
locality-sensitve hash functions. For \emph{cosine similarity}, we use the
following hash function based on random projections:
\begin{align*}
  \text{hash}(\vec{v}) &= \text{sgn}(\vec{v} \cdot \vec{r}), \\
  &\text{where \(\vec{v}\) is the input vector} \\
  &\text{and \(\vec{r}\) is a random projection.}
\end{align*}

