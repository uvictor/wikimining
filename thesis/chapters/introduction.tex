\chapter{Introduction}

As a result of the volume of content that exists today on the internet, it has
become increasingly harder for content creators and consumers alike to manage
this content in a centralised manner. One such example is Wikipedia which is
very hard to analyse and digest at a general level, mainly because of its
prohibitive large size. We aim to discover what are the most important
Wikipedia articles and how articles evolve in popularity and influence over
time. At a smaller scale, we investigate these same problems on specific
categories.

Attempts to analyse Wikipedia have also been made in the past: quantitative
measures \cite{voss2005measuring}, semantic coverage
\cite{holloway2007analyzing}, but these studies were made back when Wikipedia
was a lot smaller than it is today and none of these papers deal with detecting
influential articles (using submodular function maximisation). More often
Wikipedia was used in natural language processing, especially for computing
semantic relatedness based on the Wikipedia categories
\cite{gabrilovich2007computing}, but these kind of papers deal only indirectly
with analysing Wikipedia itself.

We define our problem as follows: given a very large corpus of documents --
such as Wikipedia -- find a way to pick the most representative articles that
best encompass the most important topics. One of the main challenges is that
the nature of the problem is subjective because the notion of importance varies
from person to person. Some more concrete objectives are to find popular
articles -- how visited, how interlinked a page is -- or debated articles --
how changed the articles is or, in the case of Wikipedia, how many revisions
does it have.

We can interpret the problem defined in the previous paragraph as a variant of
multi-document summarisation, but its setting is different in various regards.
First of all, the corpus is much bigger than the classical datasets.
Secondly, the number of selected documents is very small -- only
\emph{\(1/10000\)th or smaller} fraction of the whole corpus. For Wikipedia, we
select between \(40\) and \(130\) articles from over {\(1.3\) million}
human-written pages. We call the problem of selecting very few documents from
an extremely large corpus: \emph{massive corpus summarisation}.  In order to
study the evolution of the Wikipedia network and find representative subsets,
we investigate different solutions using submodular function maximisation. We
adapt existing submodular functions from the literature and define new
functions to solve this problem.

Another challenge is scaling our algorithms to deal with such a large dataset.
We devise a framework -- WikiMining -- that can scale up to Wikipedia’s size,
using the Distributed Submodular Maximisation (GreeDi) protocol
\cite{mirzasoleiman2013distributed} over MapReduce \cite{dean2008mapreduce}. To
compare the different approaches we devise a simple procedure to evaluate the
quality of my algorithms.

\subsubsection{Contributions}

We have the following main contributions:
\begin{itemize}
  \item We define novel monotone submodular functions, that capture
  \emph{document importance}, to summarise massive corpora better than existing
  functions;
  \item We create a framework that scales to millions of documents, using
  GreeDi \cite{mirzasoleiman2013distributed} and MapReduce
  \cite{dean2008mapreduce};
  \item We extend, for the first time, multi-document summarisation to a
  massive corpora -- such as Wikipedia -- using submodular function
  maximisation;
  \item We use simple evaluation metrics to cross-check the different
  submodular functions;
\end{itemize}

